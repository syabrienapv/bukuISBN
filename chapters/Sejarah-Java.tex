\section{Sejarah Java}
Java adalah bahasa pemrograman yang dapat dijalankan di berbagai komputer termasuk telepon genggam. Bahasa ini awalnya dibuat oleh James Gosling saat masih bergabung di Sun Microsystems saat ini merupakan bagian dari Oracle dan dirilis tahun 1995. Bahasa ini banyak mengadopsi sintaksis yang terdapat pada C dan C++ namun dengan sintaksis model objek yang lebih sederhana serta dukungan rutin-rutin aras bawah yang minimal. Aplikasi-aplikasi berbasis java umumnya dikompilasi ke dalam p-code (bytecode) dan dapat dijalankan pada berbagai Mesin Virtual Java (JVM). Java merupakan bahasa pemrograman yang bersifat umum/non-spesifik (general purpose), dan secara khusus didisain untuk memanfaatkan dependensi implementasi seminimal mungkin. Karena fungsionalitasnya yang memungkinkan aplikasi java mampu berjalan di beberapa platform sistem operasi yang berbeda, java dikenal pula dengan slogannya, "Tulis sekali, jalankan di mana pun". Saat ini java merupakan bahasa pemrograman yang paling populer digunakan, dan secara luas dimanfaatkan dalam pengembangan berbagai jenis perangkat lunak aplikasi ataupun aplikasi. 

Java dikembangkan pada tahun 1990 oleh insinyur Sun, James Gosling sebagai bahasa pemrograman yang  berperan sebagai otak untuk peralatan pintar (TV interaktif, oven serba bisa). Gosling tidak puas dengan hasil yang ia peroleh ketika menulis program dengan C++, bahasa pemrograman lain, sehingga ia mengasingkan diri di kantornya dan menulis bahasa pemrograman baru agar lebih sesuai dengan kebutuhannya.

Gosling menamakan bahasa pemograman barunya Oak, nama sebuah pohon yang bisa ia lihat dari jendela kantornya; ia kemudian menamainya Green, dan kemudian mengganti namanya menjadi Java, berasal dari kopi Jawa (Java Coffee) , yang katanya banyak dikonsumsi dalam jumlah besar oleh pencipta bahasa ini. Bahasa pemograman ini kemudian menjadi bagian dari strategi Sun untuk menghasilkan uang jutaan dolar ketika TV interaktif menjadi industri bernilai jutaan dolar. Hal itu memang masih belum terjadi hari ini, tetapi sesuatu yang benar-benar berbeda kemudian terjadi pada bahasa pemograman baru Gosling itu.

Secara kebetulan World Wide Web menjadi begitu populer, banyak kelebihan yang membuat bahasa Gosling dapat digunakan dengan baik dan cocok pada proyek maupun alat untuk adaptasi ke Web. Pengembang Sun merancang cara bagi program yang akan berjalan dengan aman dari halaman web dan memilih nama baru yang menarik untuk menemani fokus baru bahasa itu: Java.

Walaupun Java dapat digunakan untuk banyak hal, Web menyediakan tampilan yang dibutuhkan untuk menarik perhatian internasional. Seorang programmer yang menempatkan program Java pada halaman web dapat langsung diakses ke seluruh planet “Web-surfing“. Karena Java adalah teknologi pertama yang bisa menawarkan kemampuan ini, Java kemudian menjadi bahasa komputer pertama yang menerima perlakuan bagai bintang di media.

Java adalah bahasa pemrograman untuk berbagai tujuan (general purpose), bahasa pemrogramn yang concurrent, berbasis kelas, dan berorientasi objek, yang dirancang secara khusus untuk memiliki sesedikit mungkin ketergantungan dalam penerapannya. Hal ini dimaksudkan untuk memungkinkan pengembang aplikasi “write once, run anywhere” (WORA), yang berarti bahwa kode yang dijalankan pada satu platform tidak perlu dikompilasi ulang untuk di tempat lain. Java saat ini menjadi salah satu bahasa pemrograman yang paling populer digunakan, terutama untuk aplikasi web client-server, dengan 10 juta pengguna.

\subsection{Sejarah perkembangan}
%​Bahasa pemrograman Java terlahir dari The Green Project yang berjalan selama 18 bulan, dari awal tahun 1991 hingga musim panas 1992. Proyek tersebut belum menggunakan versi yang dinamakan Oak. Proyek ini dimotori oleh Patrick Naughton, Mike Sheridan, dan James Gosling, beserta sembilan pemrogram lainnya dari Sun Microsystems. Salah satu hasil proyek ini adalah maskot Duke yang dibuat oleh Joe Palrang.

Pertemuan proyek berlangsung di sebuah gedung perkantoran Sand Hill Road di Menlo Park. Sekitar musim panas 1992 proyek ini ditutup dengan menghasilkan sebuah program Java Oak pertama, yang ditujukan sebagai pengendali sebuah peralatan dengan teknologi layar sentuh (touch screen), seperti pada PDA sekarang ini. Teknologi baru ini dinamai "*7" (Star Seven).

Setelah era Star Seven selesai, sebuah anak perusahaan Tv kabel tertarik ditambah beberapa orang dari proyek The Green Project. Mereka memusatkan kegiatannya pada sebuah ruangan kantor di 100 Hamilton Avenue, Palo Alto.

Perusahaan baru ini bertambah maju: jumlah karyawan meningkat dalam waktu singkat dari 13 menjadi 70 orang. Pada rentang waktu ini juga ditetapkan pemakaian Internetsebagai medium yang menjembatani kerja dan ide di antara mereka. Pada awal tahun 1990-an, Internet masih merupakan rintisan, yang dipakai hanya di kalangan akademisi dan militer.

Mereka menjadikan perambah (browser) Mosaic sebagai landasan awal untuk membuat perambah Java pertama yang dinamai Web Runner, terinsipirasi dari film 1980-an, Blade Runner. Pada perkembangan rilis pertama, Web Runner berganti nama menjadi Hot Java.

Pada sekitar bulan Maret 1995, untuk pertama kali kode sumber Java versi 1.0a2 dibuka. Kesuksesan mereka diikuti dengan untuk pemberitaan pertama kali pada surat kabar San Jose Mercury News pada tanggal 23 Mei 1995.

Sayang terjadi perpecahan di antara mereka suatu hari pada pukul 04.00 di sebuah ruangan hotel Sheraton Palace. Tiga dari pimpinan utama proyek, Eric Schmidt dan George Paolini dari Sun Microsystems bersama Marc Andreessen, membentuk Netscape.

Nama Oak, diambil dari pohon oak yang tumbuh di depan jendela ruangan kerja "Bapak Java", James Gosling. Nama Oak ini tidak dipakai untuk versi release Java karena sebuah perangkat lunak lain sudah terdaftar dengan merek dagang tersebut, sehingga diambil nama penggantinya menjadi "Java". Nama ini diambil dari kopi murni yang digiling langsung dari biji (kopi tubruk) kesukaan Gosling. Konon kopi ini berasal dari Pulau Jawa. Jadi nama bahasa pemrograman Java tidak lain berasal dari kata Jawa (bahasa Inggris untuk Jawa adalah Java).

\subsection{Asal-Usul Nama Java}
Kopi asal Jawa (Java Coffee) terkenal bercita rasa tinggi dan salah satu jenis Arabica yang terbaik di dunia. Namun bagi James Gosling dan rekan-rekannya di Sun Microsystems, kopi yang diseduh di sebuah kafe Peet menjadi inspirasi untuk nama bahasa pemrograman komputer baru yang berhasil dikembangkan. Java menjadi pilihan menggantikan nama Oak, dari jenis pohon yang tumbuh di depan jendela ruang kerja Gosling. Greentalk adalah nama yang diperkenalkan Gosling pertama kali untuk bahasa pemrograman tersebut dengan file ekstensi ".gt" sebelum menjadi Oak.

Sayangnya nama Oak sudah dipakai perusahaan lain, yaitu Oak Technology sebagai merek dagang produknya. Usaha untuk mengganti nama ternyata tidak semudah yang dibayangkan. Atas usul pengacara dan ahli hukum perusahaan, perdebatan dengan berbagai pendapat dilakukan para insinyur, manajer pemasaran, penasehat hukum, dan direksi Sun Microsystems untuk menemukan nama yang tepat selama berhari-hari.

Nama-nama yang kemudian menjadi kandidat adalah Silk, DNA, dan Java. Entah siapa yang pertama kali mengusulkan nama Java atau sejak kapan nama Java dipakai, tidak begitu diperhatikan karena alternatif pilihan nama tersebut dilakukan secara kolektif. Kelak Kim Polese, manajer pemasaran saat itu yang sekarang adalah CEO Marimba Inc. akhirnya memakai merek dagang Java.

Kelahiran Java berawal dari ambisi Sun Microsystems untuk menciptakan platform universal yang dapat mengintegrasikan berbagai mesin. Projek rahasia yang membawa misi besar itu diberi nama Green Project. Projek tersebut melibatkan Patrick Naughton, Mike Sheridan, dan James Gosling serta kemudian dibantu 13 orang staf. Mereka bekerja secara tertutup dan mengasingkan diri pada sebuah gedung di Sand Hill Road, Menlo Park, California, AS. Projek yang dimulai pada Desember 1990 akhirnya membuahkan hasil setelah bekerja keras selama 18 bulan dan menghabiskan dana jutaan dolar AS.

Pada 3 September 1992 mereka mendemonstrasikan Star7, sebuah PDA dengan input touchscreen (layar sentuh) yang dapat menjalankan berbagai aplikasi interaktif. Termasuk menciptakan animasi Duke yang menjadi maskot Java. James Gosling dan kawan-kawan telah mengantarkan bahasa pemrograman baru (Java) yang dapat berjalan pada semua platform peranti elektronika. Perbedaan platform diatasi dengan membuat mesin virtual pada arsitektur bahasa pemrograman yang baru. Mesin virtual tersebut akan menerjemahkan kode pemrograman menjadi bahasa yang dikenali mesin apa pun. Java juga dikenal sangat andal dan memiliki sistem keamanan sendiri.

Java hadir pada momentum yang tepat saat internet dan kebutuhan aplikasi multimedia mulai berkembang. James Gosling membuktikan kehebatan Java bersama John Gage, direktur Sun Science Office saat memberikan presentasi bertajuk "Hollywood-meets-Silicon-Valley" di awal tahun 1995. Ia berhasil memperlihatkan gerakan molekul tiga dimensi di tengah-tengah layar komputer dengan menggerakkan mouse. Apalagi sejak HotJava (sebelumnya disebut WebRunner) browser internet berbasis Java siap diluncurkan sebulan kemudian. Kerjasama antara Sun Microsystems dan Netscape untuk memasang Java pada browser Netscape Communicator saat dirilis kemudian ikut mempercepat ketenaran Java.

Sejak dirilis pada 23 Mei 1995, Java segera melejit menjadi bahasa pemrograman favorit. Java menghasilkan gelombang baru dalam dunia komputasi. Apalagi Sun memberikan source code Java secara cuma-cuma melalui internet. Dengan demikian Java segera tersebar dan setiap orang dapat mencoba dan memberikan umpan balik. Respons yang diberikan para pengguna Java ikut berkontribusi memperbaiki dari versi alpha (1.0a2) hingga versi 2 pada saat ini. Keberhasilan Sun menghadirkan Java sebagai yang terdepan dalam komunikasi internet tidak lepas dari peran James Gosling, arsitek bahasa pemrograman Java.

James Gosling lahir pada tanggal 19 Mei 1956 dari tiga bersaudara di dekat Calgary, Kanada. Sejak kecil dia memang sangat tertarik dengan elektronika. Saat usia 12 tahun, orangtuanya mendapatinya berhasil membuat permainan tic tac toe dengan memanfaatkan komponen suku cadang telefon dan televisi. Melihat minat dan bakat tersebut, suatu ketika sahabat orangtuanya mengajak Gosling ke laboratorium komputer di Universitas Calgary. Saat itu usianya masih 14 tahun.

Sejak saat itulah ia lebih sering menghabiskan banyak waktu di laboratorium komputer daripada belajar di kelas. Lulus dari SMU, ia melanjutkan di Universitas Calgary. Saat menyelesaikan sarjana, ia mengembangkan editor teks Emacs, yang kelak menjadi editor teks yang paling banyak digunakan pada sistem operasi Unix. Kemudian ia mengambil pendidikan Master di Universitas Alberta sebelum melanjutkan program doktor di Universitas Carnegie Mellon di Pittsburgh. Ia memperoleh gelar Ph.D setelah berhasil mempertahankan tesisnya yang berjudul "The Algebraic Manipulation of Constraints" pada tahun 1983. Ia segera bergabung dengan IBM selepas kuliah.

Sayang hasil pekerjaannya tidak pernah diproduksi. Setahun kemudian, ia bergabung dengan Sun Microsystems hingga menjadi bagian Green Team untuk menjalankan projek rahasia Green Project. Berkat kemampuannya, kariernya segera melejit sehingga menduduki posisi Vice President (VP) Sun Microsystems dan Chief Technology Officer (CTO) Sunís Developer Product. Saat ini, ia masih berkontribusi pada Real-Time Specification of Java dan peneliti di laboratorium Sun untuk software development tools. Selain menjadi arsitek bahasa pemrograman Java, ia juga membangun sistem akuisisi data satelit, multiprosesor untuk Unix, beberapa kompiler, mail system dan insinyur utama pembuat windows manager NEWS (Network Extensible Windowing System). Akankah ia juga mengenang Pulau Jawa setiap kali menyeduh kopi panasnya di sela-sela memprogram Java? Yang jelas ia selalu senang untuk berkata, "Jika dunia berbicara dengan Inggris, internet berbicara dengan Java."
\subsection{Java}
Untuk membuat sebuah aplikasi, diperlukan bahasa pemrograman dan salah satu bahasa pemrograman populer yang terkenal tangguh adalah Java.Java sebagai salah satu bahasa pemrograman yang sudah berumur dari era 1990-an, kian berkembang dan melebarkan dominasinya di berbagai bidang. Java adalah bahasa pemrograman yang paling banyak digunakan di seluruh dunia, disusul oleh C dan C++. Java dapat digunakan untuk membuat aplikasi berbasis konsole atau text, GUI, web dan mobile device. Java bersifat platform-independence, artinya, aplikasi yang dibuat dengan Java dapat dijalankan di platform atau Sistem Operasi populer seperti Windows, Linux dan Macintosh tanpa harus merubah source code aplikasi.Selain itu Java pun menjadi pondasi bagi berbagai bahasa pemrograman seperti Kotlin, Scala, Clojure, Groovy, JRuby, Jython, dan lainnya yang memanfaatkan Java Virtual Machine sebagai rumahnya.

Java pun akrab dengan dunia saintifik dan akademik. Cukup banyak akademisi di Indonesia yang menggunakan Java sebagai alat bantu untuk menyelesaikan skripsi atau tugas akhir dengan berbagai topik yang didominasi kecerdasan buatan, data mining, enterprise architecture, aplikasi mobile, dan lainnya. Di dunia web development sendiri, Java memiliki berbagai web framework unggulan seperti Spring, Play Framework, Spark, Jakarta Struts, dan Java Server Pages.

Dapat menggunakan salah satu dari tiga IDE populer seperti NetBeans, Eclipse, atau IntellijIDEA. Java pun memiliki package manager yang mulai populer sejak digunakan di Android Studio yang bernama Gradle. Yah Java yang diciptakan oleh James Gosling ini memang diambil dari sebuah nama pulau dimana James berlibur di Indonesia. Bahkan ada beberapa package Java yang diambil dari nama - nama daerah di Indonesia seperti Jakarta Struts dan Lombok.

Struktur program Java secara umum dibagi menjadi 4 bagian:
\begin{enumerate}
    \item Deklarasi Package
    \item Impor Library
    \item Bagian Class
    \item Method Main
\end{enumerate}

Contoh :


\begin{enumerate}
\item Deklarasi Package
Package merupakan sebuah folder yang berisi sekumpulan program Java. Deklarasi package biasanya dilakukan saat membuat program atau aplikasi besar. Biasanya nama package mengikuti nama domain dari sebauh vendor yang mengeluarkan program tersebut. Aturannya: nama domain dibalik, lalu diikuti nama programnya.

\item Impor Library
Pada bagian ini melakukan impor library yang dibutuhkan pada program. Library merupakan sekumpulan class dan fungsi yang bisa kita gunakan dalam membuat program.

\item Bagian Class
Java merupakan bahasa pemrograman yang menggunakan paradigma OOP (Object Oriented Programming). Setiap program harus dibungkus di dalam class agar nanti bisa dibuat menjadi objek. Blok class dibuka dengan tanda kurung kurawal { kemudian ditutup atau diakhiri dengan }. Di dalam blok class, kita dapat mengisinya dengan method atau fungsi-fungsi dan juga variabel.

\item Method Main
Method main() atau fungsi main() merupakan blok program yang akan dieksekusi pertama kali. Ini adalah entri point dari program. Method main() wajib kita buat. Kalau tidak, maka programnya tidak akan bisa dieksekusi. Method main() memiliki parameter args[]. Parameter ini nanti akan menyimpan sebuah nilai dari argumen di command line.

\item Statement dan Ekspresi pada Java
Statement dan eksrepsi adalah bagian terkecil dalam program. Setiap statement dan ekspresi di Java, harus diakhiri dengan titik koma (;). Statemen dan ekspresi akan menjadi instruksi yang akan dikerjakan oleh komputer.

\item Blok Program Java
Blok program merupakan kumpulan dari statement dan ekspresi yang dibungkus menjadi satu. Blok program selalu dibuka dengan kurung kurawal { dan ditutup dengan }.Intinya: jika kamu menemukan kurung { dan }, maka itu adalah sebauh blok program. Blok program dapat juga berisi blok program yang lain (nested).

\item Penulisan Komentar pada Java
Komentar merupakan bagian program yang tidak akan dieksekusi oleh komputer.
Komentar biasanya digunakan untuk:
\begin{enumerate}
    \item Memberi keterangan pada kode program;
    \item Menonaktifkan fungsi tertentu;
    \item Membuat dokumentasi;
    \item dll.
\end{enumerate}
Penulisan komentar pada java, sama seperti pada bahasa C. Yaitu menggunakan:
\begin{enumerate}
    \item Garis miring ganda (//) untuk komentar satu baris;
    \item Garis miring bintang (/*...*/) untuk komentar yang lebih dari satu baris.
\end{enumerate}

\item Penulisan String dan Karakter
String merupakan kumpulan dari karakter. Kita sering mengenalnya dengan teks.

Contoh string: "Hello world"

Aturan penulisan string pada Java, harus diapit dengan tanda petik ganda seperti pada contoh di atas. Apabila diapit dengan tanda petik tunggal, maka akan menjadi sebuah karakter.

Contoh: 'Hello world'.

Jadi harap dibedakan:
\begin{enumerate}
    \item Tanda petik ganda ("...") untuk membuat string;
    \item Sedangkan tanda petik tunggal ('...') untuk membuat karakter.
\end{enumerate}

\item Case Sensitive
Java bersifat Case Sensitive, artinya huruf besar atau kapital dan huruf kecil dibedakan. Banyak pemula yang sering salah pada hal ini. Karena tidak bisa membedakan mana variabel yang menggunakan huruf besar dan mana yang menggunakan huruf kecil.
\end{enumerate}


\subsection{Mengenal Tipe Data Dasar di Java}
Berurusan dengan tipe data untuk variabel, Java memiliki sangat banyak tipe data yang dasar dan kompleks. Tipe data yang kompleks dapat Anda temukan seperti ArrayList, HashMap, HashTable, Vector, Array, dan lainnya. Untuk tipe data dasar, Anda dapat menggunakan int, float, double, String, Boolean, dan lainya. Untuk membuat sebuah array dari tipe data dasar, Anda dapat menggunakan tanda "[]" setelah mengetik tipe data yang Akan digunakan.

\subsection{Variabel}
Variabel adalah sebuah tempat untuk menampung value dimemori, dapat dimisalkan seperti sebuah ruangan atau wadah, variabel dibagi dua berdasarkan ruang lingkup yaitu variable lokal dan global, untuk menentukan variabel global atau lokal itu tergantung dari tempat dideklarasikannya variabel pada program yang sedang dibuat. Variabel global yaitu variabel yang dapat diakses di semua lingkup dalam program yang sedang dibuat, dalam kata lain variabel global ini dapat dikenali oleh semua fungsi dan prosedur, sementara variabel lokal yaitu variabel yang dapat diakses hanya di lingkup khusus, dalam kata lain variabel lokal ini hanya bisa diakses pada fungsi/prosedur dimana variabel itu dideklarasikan.Untuk mendeklarasikan sebuah variabel, Anda harus menulis terlebih dahulu tipe data variabelnya, kemudian nama variabel, dan wajib menginisialisasi variabel agar tidak error

Berikut ini akan ada beberapa para ilmuan yang memberikan pengertian variabel :
\begin{enumerate}
    \item F.N Kerlinger
Pengertian variabel menurut F.N Kerlinger merupakan suatu konsep yang memiliki macam-macam nilai dari suatu konsep yang dapat di rubah. Sehingga konsep tersebut akan mendapatkan titik kesimpulan yang tepat dan terbaik.

    \item Sutrisno Hadi
Variabel merupakan variasi dari objek penelitian, seperti tinggi badan manusia yang divariasikan dengan berat badan maupun usia yang dimiliki. Sehingga menghasilkan nilai kuantitatif dari suatu penelitian yang diterapkan secara real atau nyata.

    \item Sugiono
Pengertian Variabel dari Sugiono merupakan segala sesuatu yang diproses melalui informasi tentang suatu hal dari penelitian untuk dipelajari dan mendapatkan hasil dari penelitian tersebut. Yang mana akan ada kesimpulan dari proses penelitiannya.

    \item Freddy Rankuti
Freddy Rankuti menerapkan variabel dengan artian suatu konsep yang memiliki nilai bervariasi. Yang mana nilai tersebut dibagi menjadi 4 data yang berbeda. Seperti rasio, skala, ordinal, nominal dan internal.

    \item Suharsimi Arikunto
Variabel merupakan objek penelitian yang menjadi perhatian pada suatu titik objek penelitian. Yang nantinya akan mendapatkan nilai dari kesimpulan suatu proses.

    \item Bagja Waluya
Konsep yang tidak pernah ketinggalan dalam setiap eksperimen yang dilakukan oleh seseorang. Dari eksperimen tersebut akan menghasilkan suatu data yang berguna sebagai bukti otentik suatu penelitian.

    \item Moh. Nazir
Berikutnya, mengenai pengertiaan variabel menurut Moh. Nazir adalah suatu konsep yang memiliki bermacam-macam nilai yang nyata. Dalam suatu penelitian yang menghasilkan garis besar dari adanya nilai kualitas dan kuantitas.

    \item Sugiarto
Menurut Sugiarto variabel adalah suatu karakter yang dapat di observasi dari unit amatan yang merupakan pengenal atau atribut dari anggota kelompok. Maksud dari variabel ini adalah terjadinya proses variasi antara objek satu dengan objek yang lain. Yang mana aturan masing-masing kelompok memiliki perbedaan variasi.

    \item Tri Mutiara
Suatu proses yang berjalan dengan baik hingga mendapat perhatian dengan fokus pada pengaruh nilai yang value. Itulah pengertian variabel menurut Tri Mutiara. Yang mengartikan variabel sebagai cara terbaik mendapatkan hasil penelitian.

    \item Bhisma Murti
Definisi variabel menurut bhisma murti adalah adanya fenomena yang memiliki variasi nilai pada sebuah observasi. Yang mana variasi nilai itu dapat di kukur dengan cara kualitatif dan kuantitatif. Sehingga menghasilkan data yang benar dan tepat.

    \item Dr Ahmad Watik Pratiknya
Konsep yang memiliki variabilitas dengan penggambaran suatu abstraksi dari fenomena tertentu. Yang mana konsep tersebut berupa data seperti asal kepemilikan ciri yang bervariasi, inilah yang disebut variabel.
\end{enumerate}
\par
Berikut merupakan standar-standar dalam penulisan variabel:
\begin{enumerate}
\item Nama variabel diawali dengan huruf atau garis bawah, contoh: nama, \_nama, namaKu, nama\_variabel.
\item Karakter selanjutnya dapat berupa huruf, garis bawah atau angka, contoh: \_\_nama, nama1, p1.
\item  Nama variabel tidak boleh diawali dengan angka
\item Karakter bersifat case-sensitive (huruf besar dan huruf kecil dibedakan), contoh: Nama dan NAMA keduanya memiliki arti yang berbeda dan merupakan variabel yang berbeda.
\item Nama variabel tidak boleh menggunakan kata kunci yang ada pada bahasa pemrograman python, contoh: if, else, while
\end{enumerate}

Macam-Macam Variabel :
\begin{enumerate}
    \item Variabel dependen
Variabel dependen merupakan variabel yang tidak bebas. Mereka terikat dan mempengaruhi setiap variabel lainnya. Seperti variabel independen, yang memiliki perubahan kuat yang di timbulkan oleh variabel independen.

    \item Variabel Independen
Untuk variabel independen, pada dasarnya variabel ini akan membawa perubahan yang membawa hasil dari adanya data dalam suatu proses penelitian. Yang nantinya akan ada keterikatan antara variabel dependen.

    \item Variabel Moderator
Adanya keterkaitan suatu proses antara variabel bebas dengan variabel terikat. Yang mana hasil dari data tersebut akan semakin kuat. Sehingga proses dari kinerja penelitian dapat dianggap sukses jika data yang dihasilkan tepat.

    \item Variabel Intervening
Variabel ini memiliki beberapa pengaruh pada hubungan antara variabel terikat dengan variabel bebas. Yang mana mereka tidak bisa diamati ataupun diukur. Dan kedua variabel tersebut akan menghasilkan suatu informasi dengan cara logika ataupun analisa lainnya.

    \item Variabel Kontrol
Variabel ini adalah variabel yang dikendalikan secara konstan sehingga hubungan variabel bebas pada variabel terikat tidak berpengaruh pada faktor luar. Dari variabel ini bisa dikatakan bahwa nilai dan hasil variabel kontrol adalah nyata tidak terkait oleh media manapun.
\end{enumerate}

\section{Input dan Output}
Input \& output bertujuan agar pengguna dan program dapat berinteraksi,Perintah input() berguna untuk meminta inputan dari user, sehingga memungkinkan user untuk menginputkan data.\\
Perintah print() berguna untuk menampilkan output dari data yang diinputkan oleh user, sehingga data yang diinputkan user dapat ditampilkan ke layar.
\par
Contoh dari penggunaan input dan output adalah sebagai berikut:
\begin{lstlisting}[language=Python]
#Input yang ditujukan untuk user
nama = ”informatics research center” 

#output yang didapatkan user
print(“Halo”, nama, ”selamat datang ”)
\end{lstlisting}

\section{Operasi Aritmatika}
Python memiliki operasi aritmatika, antara lainnya seperti :
\begin{enumerate}
\item penjumlahan (+)
\item pengurangan (-)
\item perkalian (*)
\item pembagian (/)
\item sisa bagi/modulus (\%)
\item pemangkatan (**)
\end{enumerate}
Penggunaan dari simbol simbol ini sama hal nya dengan fungsi aritmatika pada umumnya.

\section{Perulangan}
Dalam membuat sebuah program, terkadang kita memerlukan satu baris atau satu blok kode yang sama secara berulang, disini fungsi perulangan dipakai sehingga kita tidak perlu menulis baris atau blok kode yang sama secara terus menerus, dalam python perulangan dibagi menjadi 2, yaitu for dan while.

\subsection{For}
For merupakan perulangan yang akan mengulang kondisi true sampai batas yang telah ditentukan,  biasanya digunakan untuk perulangan yang mana parameter pengulangannya menggunakan list atau range. Berikut ini merupakan contoh penggunakan sintaks perulangan for.
\begin{lstlisting}[language=Python]
for i in range (0 ,10): 
		print( i )
\end{lstlisting}

\subsection{While}
While merupakan perulangan yang akan terjadi apabila kondisinya True, perulangan akan terus berjalan hingga diperoleh kondisi False.erikut ini merupakan contoh penggunakan sintaks perulangan while.
\begin{lstlisting}[language=Python]
#perulangan while
hitung = 0 
while (hitung < 9): 
  	print (’hitungan ke :’, hitung) 
  	hitung = hitung + 1
 
print ("Good bye!")
\end{lstlisting}

\section{Kondisi}
Pengambilan keputusan kadang diperlukan dalam sebuah program untuk menentukan tindakan apa yang akan dilakukan sesuai dengan kondisi yang terjadi, contoh kasus misalkan ada seorang anak bernama idam, seorang manusia yang membutuhkan makan, jika idam lapar maka idam akan makan. Maka dapat dijabarkan seperti dibawah ini :\\
\\
Kondisi, jika : \\
Idam lapar \\
Maka : \\
Idam akan makan\\
Namun terkadang kondisi juga diberikan tambahan opsi sebuah kondisi tambahan, misalkan jika idam makan maka idam kenyang, namun jika tidak maka idam akan kelaparan. Penjabarannya dapat dilihat sebagai berikut :\\
\\
Kondisi, jika :\\
 Idam makan \\
 Maka : \\
 Idam akan kenyang \\
 Jika tidak : \\
 Idam akan kelaparan\\
Contoh diatas dapat ditulis dalam sintax python dengan menggunakan kondisi, pengkondisiian dalam python dibagi menjadi 4, yaitu : IF, IF ELSE, ELIF, nested IF. Berikut merupakan pembahasannya.

\subsubsection{IF}
 IF  adalah suatu struktur yang memiliki suatu perlakuan jika terjadi suatu kondisi. Akan tetapi, tidak terjadi sesuatu yang lain atau terjadi apa-apa ketika berada di dalam luar kondisi tersebut. IF hanya menjalankan satu kondisi dan menampilkan satu output. Contoh: kondisi dimana variabel a lebih besar dari variabel b, maka tampilkan hasil bahwa a lebih besar dari b.
\begin{lstlisting}[language=Python]
#if statement 
a = 330 
b = 200 
if a > a: 
	print("a lebih besar dari b")
\end{}

\subsubsection{IF ELSE}
IF ELSE digunakan apabila kondisi yang terjadi bernilai salah, maka lakukan else. Contoh: kondisi dimana variabel a lebih besar dari variabel b, maka jika b lebih besar dari a, tampiilkan hasil b lebih besar dari a, jika salah maka tampilkan a lebih besar dari pada b
\begin{lstlisting}[language=Python]
 #else 
a = 200 
b = 33 
if b > a: 
	print("b is greater than a") 
else: 
	print("a is greater than b")
\end{lstlisting}

\subsubsection{ELIF}
Kondisi ELIF merupakan suatu strktur logika majemuk yang memiliki banyak pilihan aksi terhadap berbagai kemungkinan kejadian yang terjadi. ELIF digunakan apabila kondisi pertama tidak benar maka lakukan kondisi lain (alternatif). Contoh: kondisi dimana variabel a sama dengan variabel b, maka jika b lebih besar dari a, tampiilkan hasil b lebih besar dari a, namun jika a dan b bernilai sama, maka tampilkan a sama dengan b
\begin{lstlisting}[language=Python]
#elif 
a = 33 
b = 33 
if b > a: 
 	print("b lebih besar dari a") 
elif a == b: 
 	print("a sama dengan b")
\end{lstlisting}

\subsubsection{Nested IF}
Nested if merupakan if didalam if (if bersarang), terdapat dua if didalam satu kondisi. Contoh: variabel x sama dengan 41, kondisi pertama yaitu jika x besar dari 10 maka tampilkan lebih besar dari 10, kondisi kedua yaitu jika x besar dari 20, maka tampilkan lebih besar dari 20, jika salah maka tampilkan tidak melebihi 20.
\begin{lstlisting}[language=Python]
#nested if 
x = 41
 
if x > 10: 
 print("lebih besar dari 10,") 
 if x > 20: 
	 print("lebih besar dari 20!") 
 else: 
	 print("tidak melebihi 20.")
\end{lstlisting}

\section{Error}
\begin{enumerate}
\item NameError, terjadi apabila kode mengeksekusi nama yang tidak terdefenisikan. Contoh:
\begin{lstlisting}
nama = "Dinda Majesty"
print(Nama)
\end{lstlisting}

Maka akan menghasilkan output NameError: name ’Nama’ is not defined. error ini dapat diatasi dengan mengubah variabel yang di print sesuai dengan variabel yang didefenisikan, karena penulisan pada pyton bersifat case-sensitive

\item SyntaxError, terjadi apabila kode python mengalami kesalahan saat penulisan. Contoh: menuliskan variabel yang didahului angka (1nama = ”Dinda Majesty”) maka akan muncul eror SyntaxError: invalid syntax. error ini dapat diatasi dengan memperhatikan tata cara penulisan kode pada bahasa pemrograman python.

\item Logic error merupakan kesalahan yang terjadi karena kesalahan pembacaan data pada command perintah seperti data tidak terbaca atau tidak ada, dan tidak sesuai dengan aturannya. Contoh kesalahan tipe data yaitu 
\begin{lstlisting}
a=’4’ 
b=6 

print(a+b)
\end{lstlisting}


\item TypeError, terjadi apabila kode melakukan operasi atau fungsi terhadap tipe data yang tidak sesuai. Contoh: melakukan penjumlahan terhadap tipe data string dan integer. eror ini dapat diatasi dengan mengubah tipe data string menjadi integer.
\begin{lstlisting}
a = "10"
b = 5

print(a + b)
\end{lstlisting}
Maka akan menghasilkan output eror TypeError: can only concatenate str (not ”int”) to str

\item IdentationError, terjadi apabila kode perulangan atau pengkondisian tidak menjorok kedalam (tidak menggunakan identasi), error ini dapat diatasi dengan menambahkan tab atau spasi. Contoh
\begin{lstlisting}
a = 200 
b = 330 

if b > a: 
print("b lebih besar dari a")
\end{lstlisting}
Maka akan menghasilkan output eror IndentationError: expected an indented block
\end{enumerate}

\section{Try Except}
Try Except merupakan salah satu bentuk penangan error di dalam bahasa pemrograman python, perintah try except ini memiliki fungsi untuk menangkap sebuah error dan tetap menjalankan program kita, sehingga program yang sedang dijalankan akan mengeksekusi program hingga akhir. Contohnya terdapat pada listing berikut
\begin{lstlisting}[language=Python]
a="1"
b=2

try:
	a+b
except:
	print("Error, kedua tipe data berbeda")
\end{lstlisting}